\documentclass{article}
\usepackage{listings}
\usepackage{xcolor} % For custom colors if needed


\title{Exercise 2-2}

\lstdefinelanguage{Prolog}{
    morekeywords={is, fail, true, not}, % Add more Prolog keywords here if needed
    morecomment=[l]{\%},                % Line comment with %
    morestring=[b]",                    % Strings delimited by "
    sensitive=true                      % Prolog is case-sensitive
}

\lstset{
    language=Prolog,
    basicstyle=\ttfamily\footnotesize,  % Typewriter font for code
    keywordstyle=\color{blue},          % Color for keywords
    commentstyle=\color{green},         % Color for comments
    stringstyle=\color{red},            % Color for strings
    numbers=left,                       % Line numbers on the left
    numberstyle=\tiny\color{gray},      % Style for line numbers
    stepnumber=1,                       % Number every line
    frame=single,                       % Frame around the code block
    breaklines=true,                    % Line breaks automatically
    showspaces=false,                   % Don't show spaces as special characters
    showstringspaces=false,             % Don't mark spaces in strings
    tabsize=4                           % Tab size
}



\begin{document}
\maketitle


\begin{lstlisting}[caption=Unaltered middle predicate]
% middle(X,Xs)
% X is the middle element in the list Xs
middle(X, [X]).
middle(X, [First|Xs]) :-
    append(Middle, [Last], Xs),
    middle(X, Middle).
\end{lstlisting}


\begin{lstlisting}[caption=Altered version 1]
middle(X, [First|Xs]) :-
    append(Middle, [Last], Xs),
    middle(X, Middle).
middle(X, [X]).
\end{lstlisting}

\begin{lstlisting}[caption=Altered version 2]
middle(X, [First|Xs]) :-
    middle(X, Middle),
    append(Middle, [Last], Xs).
middle(X, [X]).
\end{lstlisting}


\begin{lstlisting}[caption=Altered version 3]
middle(X, [X]).
middle(X, [First|Xs]) :-
    middle(X, Middle),
    append(Middle, [Last], Xs).
\end{lstlisting}





\end{document}
